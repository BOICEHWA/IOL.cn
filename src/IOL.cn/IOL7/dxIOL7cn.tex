\def \thisling{中文}
\def \thislang{中文}
\def \thistext{中文文本}
\def \olympiad{第七届国际语言学奥林匹克竞赛}
\def \pgheader{第七届国际语言学奥林匹克竞赛}
\def \xPoljska{波兰}
\def \yWroclaw{弗罗茨瓦夫}
\def \olydates{2009年7月26 — 31日}
\def \probindl{个人赛题目}
\def \solsindl{个人赛解答}
\def \probteam{团体赛题目}
\def \soluteam{团体赛解答}
\def \probword #1{题 \##1}
\def \pontword{分}
\def \regulats{解答规则}
\def \regulatx{毋需抄题. 将不同问题的解答分述于不同的答题纸上. 每张纸上注明题号、座位号和姓名}
\def \towarrant{否则答题纸可能被误放或遗失.}
\def \regulaty{解答需详细论证. 无解释之答案, 即便完全正确, 也会被处以低分.}
\def \editorsz{编者}
\def \edinchef{主编}
\def \goodluck{祝你好运}
\def \answersp{答案}
\def \giphratr #1{下列是汉语短语及其#1翻译}
\def \gisentra #1{下列是#1语句及其汉语翻译}
\def \fordinto #1{翻译成#1}
\def \fallways #1{\fordinto #1 (若有多种可能, 请全部写出)}
\def \fillgaps{请填补空缺}
\def \spokenca #1#2{#2, 约有#1人使用该语言}
\def \introman{拉丁字母转写中, }
\def \oshiroko #1{#1 $\approx$~汉语拼音 \word{o}}
\def \eshiroko #1{#1 $\approx$~英语\word{hat}中的\word{a}}
\def \aconsons{是辅音}
\def \jotsound #1{#1~=汉语拼音 \word{y}}
\def \djaffric #1{#1~= 英语 \word{judge} 中的 \word{j}}
\def \chaffric #1{#1~= 英语 \word{church} 中的 \word{ch}}
\def \velarnas #1{#1~= 普通话 \word{杭} (\word{háng})中的\word{ng}}
\def \retrflex #1#2#3{#1,~#2 和~#3 $\approx$ 通用北美口音英语 \word{barn}, \word{marsh} 和 \word{art} 中的 \word{b}, \word{sh} 和 \word{t}, 发音时舌尖向后卷}
\def \syllabir #1{#1~类似普通话和北方方言\word{这儿} (\word{zhèr})中的\word{er}}
\def \burmethe #1#2{#1~$\approx$ 英语\word{with}中的~#2}
\def \glotstop #1{#1~是辅音 (声门塞音)}
\def \longmark #1{标记 #1 表示长元音}
\def \nasaleng #1{#1~表示前面的元音鼻化}
\def \aspirate #1{#1~表示前面的辅音送气 (发音时有明显气流送出)}
\def \hilotone #1#2{标记 #1 和 #2 表示声调的高与低}
\def \nonemidt{若无声调符号标记, 则音节为中调}
\def \burorder{下列是24个缅甸孩童的名字及其出生日期}
\def \sixmoreb #1#2{另有6个缅甸孩童于 #1, 及 #2出生}
\def \nomchaos{他们的名字如下}
\def \whowhenb{每个孩童出生于何时}
\def \burmasol{从题中可见, 出生日期星期几相同的孩童, 其名字的首音素相似}
\def \nosunday{题中所给资料无出生于星期日的孩童, 也没有名字以元音开头的孩童}
\def \ladsnoms{男}
\def \lassnoms{女}
\def \bairname{名字}
\def \borndate{出生日期}
\def \oldindin{下列是部分古印度-雅利安语词干, 语言学家认为其保留了最为古老的 (印欧语系) 重音位置}
\def \oldindiv{这些词干分为词根和后缀, 中间用连字符连接}
\def \oldindas{阐释为何无法用题中的语料决定下列词干的重音位置}
\def \oldindan{指出下列词干的重音}
\def \wostress #1{重音元音上有~#1标记}
\def \washaqua{洗涤用水}
\def \gerontok{长老政治}
\def \hyenamot{鬣狗}
\def \hippopot{河马}
\def \slatemot{石板}
\def \bigsieve{宽口筛}
\def \bilakoro{未受割礼的男孩}
\def \surprise{突然性}
\def \epistola{信函}
\def \causemot{因为}
\def \hailword{欢迎}
\def \matchman{卖火柴者}
\def \niechmot{但愿......}
\def \arcoiris{彩虹}
\def \advertis{广告}
\def \rustword{锈}
\def \lamplite{灯}
\def \xianpope{基督教牧师}
\def \wolowolo{蠓}
\def \wolomesi{蠓蜜}
\def \jamanakE{青春的欢愉}
\def \wolfword{狼}
\def \enemymot{敌人}
\def \smokemot{烟}
\def \stepword{步}
\def \premiere{第一}
\def \ghrishvi{热情洋溢的}
\def \sleepmot{睡}
\def \cloudmot{云}
\def \nakedmot{裸体的}
\def \foamword{泡沫}
\def \lordword{王}
\def \messengr{信使}
\def \hymnword{圣歌}
\def \castrate{阉割的}
\def \murderwd{杀戮}
\def \drauxtwd{畅饮}
\def \heatword{热量}
\def \desirewd{渴望}
\def \partword{分享}
\def \soilword{土壤}
\def \movement{走}
\def \cordword{结}
\def \burdenwd{负担}
\def \praisewd{赞扬}
\def \effusion{倾泻}
\def \activewd{移动的}
\def \deadweap{致命武器}
\def \plouxing{耕地}
\def \rollword{滚动}
\def \aquaskin{皮包}
\def \prosperi{繁荣}
\def \demolixy{拆除者}
\def \horsebak{坐在马背上}
\def \mornlixt{晨光}
\def \pterocle{沙鸡}
\def \kindbird{一种鸟}
\def \euphemis{委婉语}
\def \Monday{星期一}
\def \Tueday{星期二}
\def \Wedday{星期三}
\def \Thuday{星期四}
\def \Friday{星期五}
\def \Satday{星期六}
\def \Sunday{星期日}
\def \inmanden{下列是马宁卡语和班巴拉语词语 (分别用西非书面语言与拉丁字母转写) 及其汉语翻译}
\def \kantenko{西非书面语言由几内亚启蒙家索勒梅纳 · 坎特于1949年发明}
\def \dirnkorl{西非书面语言从右到左读写}
\def \alphanko{该书写系统为全音素文字: 每个字母表示1个辅音或元音}
\def \linklett{每个词语中的字母连写}
\def \tontilde{元音字母上的波浪线表示低调, 无波浪线表示高调}
\def \tonemidd{但若某元音与其前面的字母声调标记相同 (都有或无波浪线), 该元音为中调}
\def \nkoshort{若两相邻音节元音相同, 声调表记亦相同 (根据上述规则, 两元音皆有波浪线或皆无波浪线), 则仅写出第2个元音}
\def \gmanding{班巴拉语和马宁卡语隶属于曼德语族的曼丁哥分支}
\def \maliguin{这2种语言主要使用于马里, 几内亚及其它西非国家}
\def \proxling{两者的联系非常紧密, 其区别与本题无关}
\def \burmtran{缅甸名字用简化的拉丁字母转写表示}
\def \nahuorth{纳瓦特尔语语句以简化的正字法表示}
\def \verberst{纳瓦特尔语语句由谓语起始}
\def \nounexts #1{主语和宾语 (前有定冠词 #1) 以自由语序跟随其后}
\def \nahupref{动词有下列前缀}
\def \nahusuff{和下列后缀}
\def \lengprec #1{之前的 #1 延长}
\def \chanprec #1#2{之前的 #1 变为 #2}
\def \Sb{主语}
\def \Ob{宾语}
\def \bisOb{另一个宾语}
\def \Vintrans{不及物动词}
\def \Vtransit{及物动词}
\def \suppletV{同样的动作, 及物和不及物动词常异干表示}
\def \reNahua{古典纳瓦特尔语是墨西哥阿兹特克帝国的语言}
\def \lgNahua{纳瓦特尔语}
\def \lgSulka{苏卡语}
\def \geSulka{苏卡语隶属于东巴布亚语系}
\def \eastnewb{在巴布亚新几内亚的东新不列颠省}
\def \loSulnum{以下是苏卡语数词的组成部分}
\def \meansadd{加}
\def \meansdup{2倍}
\def \singular{单数}
\def \pluralup{多数 (3个或以上)}
\def \noundiff{对这2个数, 名词有不同的形式}
\def \specnums{4个或4的倍数个椰子, 以及2个, 4个或其倍数个面包果用独立的词语表示}
\def \cocSolve #1{个椰子}
\def \betSolve #1{个槟榔果}
\def \panSolve #1{个面包果}
\def \yamSolve #1{个山药}
\def \notebetl{槟榔果实际上是某种棕榈的种子}
\def \noteyams{山药是热带植物薯蓣的可食块茎}
\def \rematole{阿托利是一种用精磨玉米粉制作的热饮}
\def \thexatol{阿托利}
\def \thexwine{酒}
\def \thexbook{书}
\def \medicins{治疗师}
\def \medicinb{为治疗师}
\def \merchans{商人}
\def \merchant{商人}
\def \carpters{木匠}
\def \carpterx{木匠}
\def \carptert{木匠}
\def \kobietat{女人}
\def \kobietax{女人}
\def \kobiecie{向女人}
\def \canzonex{歌}
\def \schlaeft{睡觉}
\def \prepares #1{准备#1}
\def \tumakunw #1{你为某人准备#1}
\def \nahuatlc{为某人打你}
\def \nahuatld{为你打某人}
\def \tikwitek{你打他}
\def \ichdrink #1{我喝#1}
\def \hedrinks{喝}
\def \umeleave #1{你为我留下#1}
\def \tikmakam #1{你向他给予#1}
\def \titzahtz{你呼喊}
\def \tucantas{你唱}
\def \egocanto #1#2{我#2唱#1}
\def \singsong #1{唱#1}
\def \iloveyou{我爱你}
\def \nisprech #1#2{我#2说#1}
\def \yoquiero #1{我想要#1}
\def \nihuetzi{我摔倒}
\def \applicat{为......做}
\def \causatif{使......}
\def \caudrink #1{使#1喝}
\def \nicaunek #1#2{我使#1想要#2}
\def \ticausat #1#2{你使#1#2}
\def \zulieben #1{爱#1}
\def \zulassen #1{留下#1}
\def \wifellun #1{女人使#1摔倒}
\def \causeste #1{使你#1}
\def \tetolass #1{留下#1}
\def \sleepste{睡觉}
\def \causesme #1{使我#1}
\def \metomake #1{准备#1}
\def \shoutsme{呼喊}
\def \hij{他}
\def \ninech{第一人称单数}
\def \timitz{第二人称单数}
\def \cuq{第三人称单数}
\def \tet{某人}
\def \te{\tet}
\def \tla{某物}
\def \wennword{若词根中的塞音}
\def \stpinrdx #1{为#1}
\def \heliline{浊音}
\def \unvoiced{清音}
\def \vwlinsfx #1{且后缀中的元音为#1}
\def \stressed #1{则重音在#1}
\def \radixloc{词根}
\def \sufixloc{后缀}
\def \astprule{仅当词根中恰含1个塞音时, 本规则适用}
\def \twonilup #1#2{若词根中有2个塞音 (#1), 或无塞音 (#2), 则无法决定重音位置}
\def \ABname{Alexander Berdichevsky}
\def \BBname{Bozhidar Bozhanov}
\def \SBname{Svetlana Burlak}
\def \DGname{Dmitry Gerasimov}
\def \XGname{Ksenia Gilyarova}
\def \IGname{Ivaylo Grozdev}
\def \SGname{Stanislav Gurevich}
\def \IDname{戴谊凡}
\def \AHname{Adam Hesterberg}
\def \BIname{Boris Iomdin}
\def \RPname{Renate Pajusalu}
\def \APname{Alexander Piperski}
\def \MRname{Maria Rubinstein}
\def \LFname{Ludmilla Fedorova}
\def \TTname{Todor Tchervenkov}
\def \MCname{Maria Cydzik}
\def \EKname{Evgenia Korovina}
\def \QCname{曹起曈}
\def \MLname{刘闽晟}
\def \edinames{\ABname, \BBname, \IDname, \LFname, \DGname, \XGname, \SGname, \AHname, \RPname, \APname, \TTname\ (\edinchef)}
\def \whowroti{\QCname, \MLname}
\def \whowrotj{\QCname, \MLname}
\def \quoted #1{“#1”}
\def \enqueten{姓名}
\def \enquetep{座位号}
\def \enquetea{你做了哪些题目?}
\def \enqueteb{你最喜欢哪道题?}
\def \enquetec{你觉得哪道题最难?}
\def \enqueted{你觉得哪道题最简单?}
\def \genViet{越南语隶属于南亚语系}
\def \specamln #1{在越南, 约有66 000 000人使用该语言}
\def \vietones{越南语有6个声调}
\def \arvowels{是元音}
\def \palatnas{$\approx$ 普通话\word{娘} (\word{niáng}) 中的 \word{ni}}
\def \retrfley #1#2#3{#1,~#2 与~#3 $\approx$ 通用北美口音英语 \word{rye}, \word{marsh} 与 \word{art} 中的 \word{r}, \word{sh} 与 \word{t}, 发音时舌尖向后卷}
\def \vietnach{$\approx$ 英语 \word{cheat} 中的 \word{cheat}}
\def \vietnadj{$\approx$ 英语 \word{jeep} 中的 \word{j}}
\def \glottalh{= 英语~\word{h}}
\def \frivelar{= 普通话 \word{喝} (\word{hē}) 中的 \word{h}}
\def \dittovoi{是其浊音}
\def \aspirath{是送气的~\word{t} (即汉语拼音 \word{t})}
\def \vietcons #1#2#3#4#5{#1 = \word{k}, #2 = \word{d}, #3 = \word{z}, #4 = \word{f}, #5 = \word{s} (清音)}
\def \marktone #1#2{其中1个声调无标记, 另5个声调用元音上方 (#1) 或下方 (#2) 的附加符表示}
\def \listfreq{下列是50个最常用的越南语单词及其在百万词语料库 (文本的集合) 中出现的频数}
\def \tranmost{以下是选自高阶初学者越南语教程前10课的课文, 请尽可能多地翻译之}
\def \allyfive{上述词语, 除5个之外, 皆出现在课文中}
\def \hixlited{这些词语在文中予以高亮标出}
\def \alphword{下列是50个最常用词语中见于课文者, 依字母表顺序排列}
\def \seelemap{地处中国以南}
\def \texttita{我的房间}
\def \texttitb{南先生在河内大学学习韩语}
\def \texttitc{李先生来越南}
\def \texttitd{文雄为\emph{Offo}公司工作}
\def \texttite{我的家人}
\def \texttitf{我住在胡志明市}
\def \texttitg{餐馆}
\def \texttith{化市\emph{(}顺化\emph{)}的纪念品店}
\def \texttiti{去越南的票}
\def \texttitj{金星 \emph{(Sao Mai)} 宾馆}
