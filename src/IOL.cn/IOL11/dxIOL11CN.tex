\def \thistext{中文文本}
\def \nthIOL #1{#1国际语言学奥林匹克竞赛}
\def \thisth{第十一届}
\def \thisland{英国}
\def \thislandd{大不列颠及北爱尔兰联合王国}
\def \thistown{曼彻斯特}
\def \Julyname{7月}
\def \Auguname{8月}
\def \olydates #1#2#3#4{#4年#3#1 — #2日}
\def \leafword #1{答题纸 \##1}
\def \probword #1{题 \##1}
\def \probsing #1{#1 题目}
\def \probplur #1{#1 题目}
\def \respsing #1{#1 答案}
\def \solusing #1{#1 解答}
\def \soluplur #1{#1 解答}
\def \indicont{个人赛}
\def \teamcont{团体赛}
\def \Teamword{团队}
\def \pontword{分}
\def \regulats{解题规则}
\def \regulado{毋需抄题}
\def \regulare{将不同问题的解答分述于不同的答题纸上}
\def \regulami{每张纸上注明题号、座位号和姓名}
\def \towarrant{否则答题纸可能被误放或遗失}
\def \regulaty{解答需详细论证}
\def \regulatz{无解释之答案, 即便完全正确, 也会被处以低分}
\def \editorsz{编者}
\def \edinchef{主编}
\def \rulesmot{规则}
\def \answersp{答案}
\def \enquetex{问卷}
\def \enqueten{姓名}
\def \enquetep{座位号}
\def \enquetea{你做了哪些题目}
\def \enqueteb{你最喜欢哪道题}
\def \enquetec #1{你觉得哪道题#1}
\def \seemhard{最难}
\def \seemeasy{最简单}
\def \goodluck{祝你好运}
\def \quoted #1{“#1”}
\def \transcri{写下下面的单词和短语是如何发音的}
\def \giphratr #1{下列是汉语短语及其#1翻译}
\def \giwordss #1{下面是一些#1单词和短语}
\def \givemots #1{下面是一些#1单词}
\def \givesent #1{下面是一些#1语句}
\def \inthelg #1{#1}
\def \Muna{穆纳语}
\def \Piraha{毗拉哈语}
\def \Yidiny{伊帝尼语}
\def \Yukaghir{尤卡吉尔语}
\def \inTundra #1{苔原#1}
\def \andtrans #1{及其#1翻译}
\def \siscript{它们在正常语气下的发音的简化抄写}
\def \chaotict{(乱序排列)}
\def \madelist #1{十五年前, 英国诗人, 评论家和传记作者#1编译了一张表}
\def \ginNuskh #1{下面两页纸包含了这张表的格鲁吉亚语翻译, 使用古老的小草体字母书写 (九世纪)}
\def \tranmost #1{将其尽可能多的翻译成#1}
\def \fordinto #1{翻译成#1}
\def \tothislang{中文}
\def \tolgsky #1{#1}
\def \tolg #1{#1}
\def \Thelg #1{#1}
\def \infamily #1#2{#1属于#2语系}
\def \toAustNs{南岛}
\def \toPNyfam{帕马-恩永甘}
\def \spokenca #1#2{#2, 大约 #1 人使用该语言}
\def \inInesia{在印度尼西亚}
\def \inQueend{在澳大利亚昆士兰州}
\def \piragena #1{#1是巴西亚马孙的孤立的毗拉哈人的土著语言}
\def \soloMura{它是穆拉语系唯一一个未消亡的语言}
\def \onyukaga #1{苔原#1是西伯利亚东北部一个小民族的语言}
\def \onyukagb{只有几十位年长的老人使用该语言}
\def \onyukagc #1#2{因为许多尤卡吉尔人已经抛弃了它, 改为俄语或者他们的人数更多的邻居所使用的语言的一种, 后者往往成为更为先进的物质文化的传递者#2}
\def \egYakut{比如雅库特人}
\def \ifnosure{如果你不确定如何翻译其中一些单词, 请解释理由}
\def \motmeans #1#2{单词 #1 表示 #2}
\def \twomsame{这些单词中有两个有相同的意思, 数据中也有两个单词有相同的意思}
\def \CMUniloc{卡内基·梅隆大学}
\def \countUSA{美国}
\def \villPitt{匹兹堡}
\def \telepata #1#2{在 #1 年#2举行的一系列实验中, 研究人员首先给志愿者展示一些英文单词, 同时记录下他们脑部不同区域的活动情况}
\def \location #1{区域 #1}
\def \activite #1#2{区域 #1 #2}
\def \activaby #1{被#1激活}
\def \longmots{长单词}
\def \actibyth #1{\activaby {#1的念头}}
\def \shelterO{庇护所}
\def \shelter{庇护所}
\def \manipulO{操纵}
\def \manipul{操纵}
\def \feedingO{吃}
\def \feeding{吃}
\def \high{高}
\def \low{低}
\def \telepatc #1{然后志愿者被要求去想一个预定的 #1 词列表中的其他一些单词, 并被再次测量大脑活动的情况}
\def \telepats #1{研究人员声称前三个因素有较高的生态效度 #1 和生存值}
\def \telepati{即, 实验结果符合真实生活中人类行为的数据}
\def \telepatt #1{而区域 #1 负责印出的词的低级视觉表示}
\def \telepate{下面你可以找到四个脑部区域的活跃等级的数据, 对应志愿者所想的单词}
\def \telepatd{借助获得的数据, 研究人员能成功地判断志愿者所想的单词}
\def \sameinfo{志愿者所想的另外六个单词的相同信息在下面给出}
%\def \explasol{Explain your solution}
\def \corrcorr{找出正确的对应关系}
%\def \litt{lit.}
\def \joqol{雅库特人}
\def \joqon #1{雅库特#10}
\def \Aj #1{\relax}
\def \saan #1{木#10}
\def \alter #1{另\an{#10}}
\def \xogiaI{大}
\def \xogiai #1{大#10}
\def \toio #1{老#10}
\def \xitaixi #1{重#10}
\def \baihiigi #1{慢#10}
\def \xaibogi #1{快#10}
\def \sabi #1{愤怒的#10}
\def \hoigi #1{脏#10}
\def \genitive{genitive}
\def \nnigen #1{\nni{#1}}
\def \nni #1{\an{#10}的}
\def \mmu #1{来自\an{#10}}
\def \ggu #1{为了\an{#10}}
\def \mujay #1{和\an{#10}}
\def \gimbal #1{没有\an{#10}}
\def \waval #1{回旋镖}
\def \mujam #1{母亲}
\def \mugaRu #1{渔网}
\def \majur #1{青蛙}
\def \bunya #1{女人}
\def \Man #1{男人}
\def \mulari #1{被接纳的人}
\def \gajagimba #1{白人}
\def \bama #1{人}
\def \judulu #1{鸽子}
\def \muyubara #1{陌生人}
\def \bimbi #1{父亲}
\def \galika #1{狗}
\def \binyjin #1{大黄蜂}
\def \bajigal #1{陆龟}
\def \uo{小孩}
\def \uoduo{外孙}
\def \aariinmyver{枪击}
\def \aariismyver{步枪的雷声}
\def \myver{雷声}
\def \saal{木头}
\def \aarii{步枪}
\def \aariinjohul{步枪的口鼻}
\def \aariidovoj{步枪套}
\def \johudawur{鼻套}
\def \yohudawur{鼻套用于给鼻子保暖}
\def \johul{鼻子}
\def \uodawur{摇篮}
\def \ewceB{点}
\def \ewceA{尖端}
\def \cohojedewce{刀刃的尖端}
\def \johudewce{鼻尖}
\def \joqodile{马}
\def \ile #1{鹿}
\def \ilennime{鹿群}
\def \ilesnime{鹿的房子}
\def \ilenlegul{鹿饲料}
\def \legul{食物}
\def \legudovoj{装供给品的大口袋}
\def \ovoj{包}
\def \cuo{铁}
\def \cuon #1{铁#1}
\def \cireme{鸟}
\def \ciremennime{巢}
\def \johunmyver{打鼾声}
\def \airplane{飞机}
\def \apartment{公寓}
\def \corn{玉米}
\def \lettuce{生菜}
\def \cup{杯子}
\def \igloo{冰屋}
\def \key{钥匙}
\def \screwdriver{螺丝刀}
\def \bed{床}
\def \butterfly{蝴蝶}
\def \cat{猫}
\def \cow{牛}
\def \abeillex{蜜蜂}
\def \refrigerator{冰箱}
\def \spoon{勺子}
\def \wordword{单词}
\def \transion{翻译}
\def \undernom{有下划线的名字属于故事里的人物}
\def \adhiadhini{\CJKunderline{恶魔}}
\def \dhini{这头恶魔}
\def \dhinihi{这群恶魔}
\def \lambuku{我的房屋}
\def \munasubc{这群蚂蚁的房屋}
\def \lagahi{这群蚂蚁}
\def \lagahino{他的那群蚂蚁}
\def \alaalaga{\CJKunderline{蚂蚁}}
\def \munasubj{我的小学生的蚂蚁}
\def \murihi{这群小学生}
\def \murihino #1{#1的那群小学生}
\def \molohino #1{#1的群山}
\def \munasubn{我女人们的那群猴子}
\def \robhinehi{这群女人}
\def \robhineno{他的女人}
\def \kontuhi{这些石头}
\def \damanaka{将变得暖和}
\def \dofanaka{很暖和}
\def \dokodoho{在远方}
\def \nakudomoho{将在远方}
\def \munazina{正在爬山}
\def \munazinm{正在去找\CJKunderline{恶魔}}
\def \munazini{正在回到那些房屋}
\def \munazink{正在砍他们的山}
\def \munazinp{正在买我的那群蚂蚁}
\def \munazine{正在吃我的那群猴子}
\def \munazinf{将买\CJKunderline{恶魔}的房子}
\def \munazinl{将爬上这位小学生的石头}
\def \munazinn{将切开我的这些香蕉}
\def \munazind{将吃掉这位女人的香蕉}
\def \munazinq{将回到那群女人的小学生身旁}
\def \munazinh{将去找\CJKunderline{猴子}}
\def \andoandoke{\CJKunderline{猴子}}
\def \demonkey{猴子}
\def \monkey #1{猴子}
\def \nime #1{房屋}
\def \cohoje #1{刀}
\def \saadovoj #1{盒子}
\def \bahoigatoi #1{猪}
\def \hixi #1{耗子}
\def \kagahoaogii #1{番木瓜}
\def \kahai #1{箭}
\def \kaoaibogi #1{雨林之魂}
\def \poogaihiai #1{香蕉}
\def \xabagi #1{巨嘴鸟}
\def \jaguar #1{美洲豹}
\def \baagiso #1{许多#11}
\def \bagiaibaabi{坏小偷}
\def \bigy{地}
\def \kagihi{黄蜂}
\def \kapiigaiitoii{铅笔}
\def \koxopa{胃}
\def \piahaogixisoaipi{用于烹制的香蕉}
\def \tagasaga{大砍刀}
\def \xaaibi{细}
\def \xiga{硬}
\def \xaapisi{胳膊}
\def \xiiaapisi{袖子}
\def \xisipoai{翅膀}
\def \xaogii{外国女人}
\def \xibogi{牛奶}
\def \xisitaixagai{弯曲的羽毛}
\def \xagai{弯曲的}
\def \xisoobai{水獭}
\def \xitiixisi{鱼}
\def \giisai{这只}
\def \persnumb #1#2{第\ifcase #1\or 一\or 二\or 三\fi 人称#2}
\def \Sg{单数}
\def \Pl{复数}
\def \singular{单数形式}
\def \plural{复数形式}
\def \Anim{anim}
\def \Inan{inan}
\def \Prs{现在}
\def \Fut{将来}
\def \carrepla #1#2{如果词根的第一个声音是#1, 它将被#2替换}
\def \elsinfix #1{否则#1将被插入在第一个辅音后}
\def \directiP #1{介词#1标示运动的方向}
\def \attribum{被修饰}
\def \attribut{修饰词}
\def \pronoone{一个名词及其后的修饰词发作一个单词}
\def \butcross #1#2{但是在短语的第一个词的尾部元音后的 #1 会消失, 并且第二个词的词首的 #2 也会消失}
\def \fromwend{音节划分从词尾开始}
\def \Sywexier{音节权重等级}
\def \primrule{一个单词最后三个音节中, 最靠右的最重的类型的音节为主重音}
\def \secorula{如果一个短语最后三个音节不包含第一个词的任意一部分, 那么该短语有次重音}
\def \secorule{它的放置与主重音规则一致, 但无视最后三个音节}
\def \iftotpar #1#2{如果#1的音节数是#2}
\def \wordLoc{单词}
\def \stemLoc{词干}
\def \stemNom{词干}
\def \radical{词根}
\def \radicar{第一个词根音节}
\def \endingN{词尾}
\def \allshort{所有的音节为短音节}
\def \sodo{偶数}
\def \liho{奇数}
\def \lgulpart{词干中最后一个偶数音节会延长}
\def \thatleng #1{词干的#1音节会延长}
\def \ultimate{最后一个}
\def \penultim{倒数第二个}
\def \folloses #1{如果词尾 #1 跟着一个长元音, 那么词尾将失去自己的元音}
\def \wordsord{语序是}
\def \sestruct{句子有如下结构}
\def \structNN{复合名词有如下结构}
\def \Propname{专有名词}
\def \nounword{名词}
\def \adjectif{形容词}
\def \verbword{动词}
%\def \Pred{predicate}
\def \Sb{主语}
\def \Ob{宾语}
\def \Posson{拥有}
\def \Possor{拥有者}
\def \Possum{被拥有}
\def \artimuna #1{如果主语既不被拥有也不是专有名词, 它的前面将有一个冠词 #1}
\def \simotend #1{如果单词以 #1 结尾}
\def \byavowel{\vocal}
\def \byacsant{\const}
\def \vocal{元音}
\def \const{辅音}
\def \prevocal{在元音前}
\def \preconst{在辅音前}
\def \voiced #1{浊#1}
\def \unvoiced #1{清#1}
\def \isaconst{是一个辅音}
\def \aconsons{是辅音}
\def \syllable{音节}
\def \twosylla{两个音节}
\def \trosylla{超过两个音节}
%\def \wordlast{the final sound of the word}
\def \islost{丢失}
\def \glotfric{英语 \word{hat} 中的 \word h}
\def \glotstop{即声门塞音}
\def \hetteken #1{标记~#1}
\def \markslen{表示元音长度}
\def \marpslen{标示其前面的元音是长元音}
\def \syllbond{表示音节边界}
\def \stremark #1{位于一个音节前标示着#1重音}
\def \primary{主}
\def \secondary{次}
\def \ifthEone{若存在}
\def \ifthEany{若存在}
\def \marklong{标出长元音}
\def \et #1{和#1}
\def \ett #1{且#1}
\def \ab{或}
\def \au{或}
\def \like{如}
\def \APname{Alexander Piperski}
\def \BIname{Boris Iomdin}
\def \DGname{Dmitry Gerasimov}
\def \BBname{Bozhidar Bozhanov}
\def \TTname{Todor Tchervenkov}
\def \IDname{戴谊凡}
\def \PSname{Pavel Sofroniev}
\def \XGname{Ksenia Gilyarova}
\def \SGname{Stanislav Gurevich}
\def \LFname{Liudmila Fedorova}
\def \SBname{Svetlana Burlak}
\def \MRname{Maria Rubinstein}
\def \ABname{Aleksandrs Berdičevskis}
\def \LPname{Aleksejs Peguševs}
\def \ASname{Artūrs Semeņuks}
\def \DRname{Daniel Rucki}
\def \MSSname{马丁·西摩·史密斯}
\def \BLname{Bruno L’Astorina}
\def \HDname{Hugh Dobbs}
\def \GHname{Gabrijela Hladnik}
\def \RSname{Rosina Savisaar}
\def \JLname{李在揆}
\def \MLname{刘闽晟}
\def \QCname{曹起曈}
\def \edinames{\SBname, \IDname, \HDname, \DGname, \XGname, \SGname\ \zagrad {\edinchef}, \GHname, \BIname, \BLname, \JLname, \LPname, \MRname, \DRname, \RSname, \ASname, \PSname, \TTname}
\def \whowroti{\MLname}
