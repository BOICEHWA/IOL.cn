\addtolength{\oddsidemargin}{-.875in}
\addtolength{\evensidemargin}{-.875in}
\addtolength{\topmargin}{-.525in}
\addtolength{\textwidth}{1.75in}
\addtolength{\textheight}{1.4in}
%\usepackage{colortbl}
\input ean13

\def\EANscan#1{\if#1-\let\next=\EANscan \else
  \advance\numdigit by1
  \ifnum\numdigit<13
  \if#1X\else \ifodd\numdigit \advance\oddsum by #1 \else \advance\evensum by #1 \fi\fi
  \let\next=\EANscan
  \ifnum\numdigit=1 \if#1X\def\tabs{\A\A\A\B\B\B}\else \settables#1\fi\def\firstdigit{#1}\else
  \ifnum\numdigit<8 \usetabAB#1\edef\frontdigits{\frontdigits#1}\else
  \ifnum\numdigit=8 \insertseparator \A \usetabC #1\def\enddigits{#1}%
  \else \usetabC#1\edef\enddigits{\enddigits#1}%
  \fi\fi\fi
  \else \testchecksum#1\usetabC#1\edef\enddigits{\enddigits#1}%
  \let\next=\EANclose
  \fi\fi \next}

\newcount \Numlines
\newcount \Width
\def \Grid {\begingroup \Gridscan}
\def \Gridscan #1{%
  \wlog{Gridscan: #1}%
  \ifnum#1=0\let \next=\Gridend
  \else \advance \numdigit by1%
  \ifodd \numdigit\else
  \Numlines=\numlines \multiply \Numlines by3%
  \Width=#1\multiply \Width by3%
  \put(\Numlines,8){\mbox{\vrule height 32pt width \Width pt depth 0pt}}\fi
  \let \next=\Gridscan
  \advance \numlines by#1%
  \fi \next}
\def \Gridend #1{\endgroup}

\def \GridEAN #1#2{
  \begin{picture}(285,48)
  \put(0,0){\framebox(285,48)}
  \put(0,8){\line(1,0){285}}
  \put(0,40){\line(1,0){285}}
  \multiput(3,0)(3,0){95}{\line(0,1){48}}
  \if 1#1%
  \put(0,0){\mbox{\vrule height 8pt width 3pt depth 0pt}}
  \put(6,0){\mbox{\vrule height 8pt width 3pt depth 0pt}}
  \put(138,0){\mbox{\vrule height 8pt width 3pt depth 0pt}}
  \put(144,0){\mbox{\vrule height 8pt width 3pt depth 0pt}}
  \put(276,0){\mbox{\vrule height 8pt width 3pt depth 0pt}}
  \put(282,0){\mbox{\vrule height 8pt width 3pt depth 0pt}}
  \fi
  \numdigit=0\numlines=0\Grid #2;
  \end{picture}}

\newcommand \codeline [6]{#1 & #2 & #3#4#5#6 & #6#5#4#3 & #3#4#5#6}
%\let\settabnew\settables
%\def\settables#1{\if#1X \def\tabs{\A\A\A\B\B\B}\else \settabnew#1\fi}

\newenvironment{assgts}{%
\renewcommand \labelenumi {\bfseries (\alph {enumi})}%
\renewcommand \labelenumii {\arabic {enumii}.}%
\begin{enumerate}}{\end{enumerate}}

\def \itemInsx {\item [\textbf{(+)}]\Insights\ (\writback)}
\def \yumi {\me${}_{1+2}$}
\newcommand \by [1]{%
\unskip\nobreak\hfil\penalty50
\hbox{}\nobreak\hfill{\hbox{\qquad\itshape #1}}\par}
\newcommand \book[1]{{\slshape #1\/}}
\newcommand \bord [1]{{\bfseries\itshape\selectfont #1\/}}
\newcommand \farord [1]{{\fontencoding{T1}\bfseries\itshape\selectfont #1\/}}
\newcommand \word [1]{{\itshape #1\/}}
\newcommand \ipaword [1]{[\ipa{#1}]}

% IPA Input
% XeLaTeX has built-in Unicode support, which means IPA input becomes a simple font selection problem.

\newfontfamily \IPAFont{CMUSerif-Roman}
\newcommand \ipa [1]{\bgroup\IPAFont #1\egroup}
\newcommand \bipa [1]{\bgroup\bfseries\IPAFont #1\egroup}

% The following is old LaTeX code.
% \newcommand \ipa [1]{\bgroup\fontencoding{T3}\selectfont\SetUnicodeOption{tipa}#1\egroup}
% \newcommand \bipa [1]{\bgroup\fontencoding{T3}\bfseries\selectfont\SetUnicodeOption{tipa}#1\egroup}
% IPA Input


\newcommand \vailien[2]{\bipa{#1\v{ɛ} á #2\v{ɛ}}}
\newcommand \vainali[2]{\bipa{#1\v{ɛ} #2}}
\newcommand \vaiphra[1]{\bipa{#1\v{ɛ}}}
\def \lOO {l\'{ɔ}\'{ɔ}}
\def \pregaffd {\underline{\hspace*{4in}}}
\def \respaffd {--- \eolength {\bord e \et\ \bord o}, \barnomac}
\def \Katla {a\-ma\-ta\-ra\-ja\-b{\hh}a\-na\-sa\-la\-ga}
\newcommand \tallstrut {\vrule height 9pt depth 0pt width 0pt}
\newcommand \deepstrut {\vrule height 0pt depth 2pt width 0pt}
\newcommand \placehold {\fbox{\fbox{\quad} \fbox{\quad} \fbox{\quad}}}
\def \hh {\textsuperscript h}
%\let \visiblespace \textvisiblespace
\newcommand \latehta [1]{\quoted{#1\space}}
\newlength\halftext
\halftext=\textwidth \advance \halftext by-50pt \divide \halftext by2
\newcommand \NB {{\ooalign{\hfil\raise.2ex\hbox{\bfseries!}\hfil\crcr\Large$\bigtriangleup$}}}
%\newcommand \upsens [2]{\MakeUppercase {#1#2}}
%\newcommand \thead {\expandafter \upsens }
\let \thead \relax

\newcommand \anshints [1]{\section*{\ansminut {#1}}
  \Teamword: \underline{\hspace*{2.4in}}
  \begin{assgts}
  \item \whataffd?\par \ifnum 135<#1\respaffd \else \pregaffd \fi.
  \item \begin{itemize}
  \item \Delwords:
  \ifnum 165<#1\bord{jarau}\else \underline{\hspace*{2.4in}}\fi, \underline{\hspace*{2.4in}}.
  \item \dLetters:
  \begin{enumerate}
  \item \ifnum 75<#1\lastlett {\bord g}\inLine 1 \wasfirst{\underline{\hspace*{0.4in}}}.\fi
  \item \item 
  \end{enumerate}
  \item \Addmacra:
  \begin{enumerate}
  \item \teislett {\bord a}\inWord {\bord{indravājrā}}\inLine 9.
  \item \ifnum 45<#1\thatlett {\bord a}\inSyll {12}\inLine 2.\fi
  \item \item 
  \end{enumerate}
  \item \Delmacra:
  \begin{enumerate}
  \item \thatlett {\bord a}\inWord {\bord{syad}}\inLine 9.
  \item \ifnum 105<#1\thatlett {\bord a}\inSyll 2 \inLine 7.\fi
  \item \item \item 
  \end{enumerate}
  \end{itemize}
  \item \whatsyli {\emph{guru}}{\bord{y\Katla m}}?
  \itemInsx:
  \end{assgts}
%
  \ifnum 45<#1\vfill \defSguru {\word{guru}}. \divindiv. \sylladiv{VCV}{V-CV}{VCCV}{VC-CV}.
  \ifnum 75<#1\par \selfdesc.
  \ifnum 105<#1\par \descmetr {\bord{y\Katla(m)}}.
  \ifnum 135<#1\par \Twosylls\ \bord{la} \et\ \bord{ga} \standfor {\bord{lag{\hh}u} \et\ \bord{guru}}.
  \mnemexpl {\bord{y\Katla (m)}}{\word{lag{\hh}u} \au\ \word{guru}}.
  \ifnum 165<#1\par \mnemexpn 8{\bord{y\Katla m}}{\emph{lag{\hh}u} \et\ \emph{guru}} (\macresto).
  \fi\fi\fi\fi\fi}

\makeatletter
\def \ps@somestyle {
\let\@oddfoot\@empty
\def\@oddhead{\textsl {\small \begin{tabular}[t]{@{}l}\olympiad\ (2011).\\ \chapname \end{tabular}}\hfill \thepage}}
%\makeatother
%\makeatletter
\def \ps@answersh {
\def\@oddfoot{{\tiny \sparecop.\hfill}}
\def\@oddhead{{\large \itshape \shortstack[l]{\enqueten:\medskip\\ \enquetep:}\hfill \probword {5}\hfill \leafword {\underline {\qquad}}}}}
\makeatother

\def \problem {\stepcounter {section}\paragraph{\probword \thesection\ (20 \pontword).}}
\def \solution {\stepcounter {section}\paragraph{\probword \thesection.}}

\newcommand \makepart [1]{\newpage
  \begin{center}%
  {\LARGE \olympiad \par }
  \vskip 1em{\begin{tabular}[t]{c}\Large \yvillage\ (\xcountry),
  \olydates {24}{31}{\Julyname}{2011}
  \end{tabular}\par }
  \vskip 1em{\large #1}\end{center}\par \vskip .5em
  \def \chapname {#1}\setcounter {section}0\setcounter {page}1}

\begin{document}
\ifx\enumLat\undefined\else\enumLat\fi

\thispagestyle{empty}
\makepart{\regulats}
\regulado. \regulare. \regulami. \towarrant.

\regulaty. \regulatz.
\vfill
\begin{center}
\textbf{\editorsz:} \edinames.\medskip

\textbf{\thistext:} \whowroti.\medskip

\large \goodluck!
\end{center}

\makepart{\probindl}
\thispagestyle{empty}
\problem \introMez\andtrans:\medskip

\centerline{\begin{tabular}{|l|l|}\hline
\textbf{kewǣpeqtaq} & \yumi\ \kewAEpeqtaq \\
\textbf{kawāham} & \ta\kawam {\mezah} \\
\textbf{nepītohnæm} & \mi\nepItohnaem \\
\textbf{kēskenam} & \ta\kEskam {\mezen} \\
\textbf{pahkǣsam} & \ta\pahkAEsam \\
\textbf{kekǣtohnæq} & \yumi\ \kekAEtohnaeq \\
\textbf{pītenam} & \ta\pItam \\
%\textbf{pītenam} & \ta\pItam {\mezen} \\
\textbf{kewǣpānæhkæq} & \yumi\ \kewAEpAnaehkaeq \\
\textbf{tawǣsam} & \ta\tawAEsam \\
\textbf{nekǣtahan} & \mi\nekAEtan {\mezah} \\
\textbf{pāhkaham} & \ta\pAhkam {\mezah} (\pAhkaheg) \\
\textbf{kekēskahtæq} & \yumi\ \kekEskahtaeq \\
\textbf{wackōhnæw} & \ta\wackOhnaew \\
\textbf{newāckesan} & \mi\newAckesan \\
\textbf{ketǣnam} & \ta\ketam {\mezen} \\
\textbf{ketāwahtæq} & \yumi\ \ketAwahtaeq \\
\textbf{wǣpohnæw} & \ta\wAEpohnaew \\
\textbf{nekāweqtam} & \mi\nekAweqtam \\
\textbf{pāhkeqtaw} & \ta\pAhkeqtaw \\
\textbf{kepītahtæq} & \yumi\ \kepItahtaed; \yumi\ \kepItahtaeg \\
\textbf{nekāwāhpem} & \mi\nekAwAhpem \\
\hline
\end{tabular}}

\begin{assgts}
\item \fordinsg {\thislang}:
\textbf{kekēskahæq}, \textbf{nepāhkenan}, \textbf{wǣpāhpew}.
\transall.
\item \fordinsg {\thelgMez}:
\begin{itemize}\setlength{\itemsep}{0pt}
\item \mi\newAEpahtan
\item \yumi\ \kekAwaeq {\mezen}
\item \ta\tawAnaehkaew
\item \ta\ketOhnaew
\end{itemize}
\end{assgts}
%
\NB\quad \rewheMez. \retheMez {5\,000--10\,000}, \elderMez, \altheMez.

\quoted{\yumi} = '\kenaq'.
\aeligatu {\textbf {æ}},
\chaffric {\textbf c}, \glotstop {\textbf q}.
\longmark {\textbf{\latehta\=}}.
%
\by{—\IDname}

\newpage


\pagestyle{somestyle}
\problem \inFaroes\andtrans:\medskip

\centerline{\begin{tabular}{|l|l|l|}\hline
\farord{bøga} & ? & \bwga \\
\farord{deyði} & \ipaword{dɛiji} & (\mi) \deyzi \\
\farord{eyður} & \ipaword{ɛijur} & \eyzur \\
\farord{glaða} & \ipaword{glɛava} & \glaza \\
\farord{gleða} & \ipaword{gleːa} & (\nad) \gleza \\
\farord{gløður} & \ipaword{gløːvur} & \glwzur \\
\farord{hugi} & \ipaword{huːwi} & \hugi \\
\farord{knoðar} & ? & (\ta) \knozar \\
\farord{koyla} & \ipaword{kɔila} & \koyla \\
\farord{kvøða} & ? & (\nad) \kvwza \\
\farord{lega} & \ipaword{leːva} & \lega \\
\farord{logi} & \ipaword{loːji} & \logi \\
\farord{løgur} & ? & \lwgur \\
\farord{móða} & \ipaword{mɔuwa} & \mOza \\
\farord{mugu} & \ipaword{muːwu} & (\nad) \mugu \\
\farord{plága} & ? & \plAga \\
\farord{ráði} & \ipaword{rɔaji} & (\mi) \rAzi \\
\farord{rúma} & \ipaword{rʉuma} & (\nad) \rUma \\
\farord{røða} & \ipaword{røːa} & (\nad) \rwza \\
\farord{skaði} & ? & \skazi \\
\farord{skógur} & \ipaword{skɔuwur} & \skOgur \\
\farord{spreiða} & \ipaword{spraija} & (\nad) \spreiza \\
\farord{søga} & \ipaword{søːva} & \swga \\
\farord{tegi} & \ipaword{teːji} & \tegi! \\
\farord{toygur} & ? & \toygur \\
\farord{tregar} & \ipaword{treːar} & (\ta) \tregar \\
\farord{trúgi} & ? & \trUgi \\
\farord{vágur} & \ipaword{vɔavur} & \vAgur \\
\farord{vegur} & \ipaword{veːvur} & (\ta) \vegur \\
\farord{viður} & \ipaword{viːjur} & \vizur \\
\farord{viga} & \ipaword{viːja} & (\nad) \viga \\
\farord{øga} & \ipaword{øːa} & (\nad) \wga \\
\hline
\end{tabular}}
%
\begin{assgts}
\item \fillgaps.
\item \descruls.
\end{assgts}
%
\NB\quad \belongsto {\ThelgFar}{\toNGerma}. \spokenca{48\,000}{\iFaroetc}.

\indetscr\jotsound {\ipaword {j}}, \wawsound {\ipaword {w}};
\ipaword {ɛ}, \ipaword {ɔ}, \ipaword {ø}, \ipaword {ʉ} \arvowels.
\longmark {\quoted{\ipa{ː}}}.
%
\by{—\APname}

\newpage
\problem \introVai\andtrans:\medskip

\centerline{\begin{tabular}{|l|l|}\hline
\vailien{kàí}{l\'{ɛ}nd\'{ɛ}} & \kais \lEndE\\
\vainali{k\'{ɔ}ánjà-lèŋ}{fǎ} & \kOanjaleNfa \\
\vailien{gbòmù}{nyìmìì} & \gbomus {\nyimii}\\
\vainali{kàí}{kàfà} & \kais {\kafa}\\
\vailien{nyìmìì jǎŋ}{gbòmù-l\'{ɛ}nd\'{ɛ}} & \nyimiijaNgbomulEndE \\
\vainali{mùsú jǎŋ}{\lOO-kàì} & \musujaNlOOkai \\
\vainali{nyìmìì kúndú}{já} & \nyimiikundus {\ja}\\
\vainali{k\'{ɔ}ánjà \lOO}{k\'{ɛ}njì} & \kOanjalOOkEnji \\
\vaiphra{kánd\'{ɔ} jǎŋ} & \kandOjaN \\
\hline
\end{tabular}}

\begin{assgts}
\item \fordinsg {\thislang}:
\begin{quote}
\vailien{mùsú}{gbòmù};
\qquad \vailien{léŋ kúndú}{nyìmìì};
\qquad \vaiphra{gbòmù-l\'{ɛ}nd\'{ɛ} kúndú}.
\end{quote}
\item \existerr {\vaiphra{kánd\'{ɔ}-l\'{ɛ}nd\'{ɛ} \lOO}}.
\corrtran {\thislang}.
\item \fordinsg {\thelgVai}:
\begin{quote}
\kOanjas {\nyimii};
\quad \leNlOOs {\ja};\\
\kaijaNlOOmusu;
\quad \nyimiileNlOO.
\end{quote}
\end{assgts}
%
\NB\quad \belongsto {\ThelgVai}{\toCMande}. \spokenca{105\,000}{\inLbrSle}.

{\bipa{ny}} \et\ {\bipa{ŋ}} \atwocons; {\bipa{ɛ}} \et\ {\bipa{ɔ}} \atwovocs.
\tonmarks {{\textbf{\latehta\'}}, {\textbf{\latehta\'}} \et\ {\textbf{\latehta\v}}}.
%
\by{—\OKname}

\bigskip

\hrule

\bigskip


\problem \inNahua\andtrans\chaotict:
%
\begin{center}
\bord{acalhuah}, \bord{achilli}, \bord{atl}, \bord{callah}, \bord{calhuah}, \bord{chilatl}, \bord{chilli}, \bord{colli}, \bord{coltzintli}, \bord{conehuah}, \bord{conehuahcapil}, \bord{conetl}, \bord{oquichconetl}, \bord{oquichhuah}, \bord{oquichtotoltzintli}, \bord{tehuah}, \bord{tetlah}, \bord{totoltetl}\bigskip

\atl, \conetl, \calhuah, \achilli, \tzin {\oquichtotolin}, \conehuah, \callah, \chilatl, \collib/\collid, \tetlah, \oquichconetl, \tehuag\ (=~\tehuaq), \chilli, \totoltetl, \acalhuah, \conehuahcapil, \oquichhuah, \tzin {\collib/\collid}\end{center}
%
\begin{assgts}
\item \corrcorr.
\item \fordinsg {\theNahua}:
\calli, \tetl, \ahuah, \tzin {\oquichtl}.
\item \fordinsg {\thislang}: \bord{cacahuatl}, \bord{cacahuatetl}, \bord{cacahuaatl}, \bord{cacahuahuah}.
\end{assgts}
%
\NB \quad \reNahua.

\cyqueska {\bord c = \bord{qu}},
\chaffric {\bord{ch}}, \wawsound {\bord {hu}},
\bord{tl} \et\ \bord{tz} \atwocons.

\achildef {\emph {Polygonum hydropiper}}.
\chiladef.
%
\by{—\LFname}

\newpage

\problem \inbarcod {EAN-13 (\au\ GTIN-13)}, \anospeak.
\subcodes {10}, \nocodnil {UPC(A)}.

\smallskip

\mbox{}
\barheight=40pt\EAN X-000000-000000%
\hfill \raise20pt\hbox{$\longrightarrow$}\hfill \GridEAN 1{111321132113211112311231123111113211321132113211321132111110}
\\
\nobarcod: \unsubcod {EAN-13}.
\zagrad{\gridrite}

\smallskip

\mbox{}
\barheight=40pt\EAN 5-000168-08555-5%
\hfill \raise20pt\hbox{$\longrightarrow$}\hfill \GridEAN 1{111321111231123222111143121111113211121312311231123112311110}
\\
\sibarcod: \issubcod 5.
%
\UKbarcod {50}. \rebarcod {5-000168}{08555}. \lastxsum.

\bigskip

\morenums: \medskip \\
\begin{tabular}{ll}
20–29 & \storefns \\
30–37 & \FRland \\
40–44 & \DEland \\
\end{tabular} \hfill
\begin{tabular}{ll}
539 & \IEland \\
64 & \FIland \\
73 & \SEland \\
\end{tabular} \hfill
\begin{tabular}{ll}
84 & \ESland \\
978 & ISBN (\books) \\
?? & \NOland \\
\end{tabular}

\begin{assgts}
\item \barcodAf {A–I}. \barcodAz:
%
\begin{enumerate}
\item \isbarcod {\bumpaper\ (\ESland)}E;
\item \smokelox\ (\IEland), \prodcode\ = 02661, \checksum\ = ?;
\item \book{\LostSymb} (\ISBNbook);
\item \farsteak\ (\storpack), \cost\ = \euros4 \et\cents{16};
\item \mopshead\ (\whence?), \fullcode\ = 4-023103-075702;
\item \chollows\ (\FIland);
\item \sirsteak\ (\storpack), \cost\ = ?; 
\item \book{Korsordboken} (\puzmagaz, \SEland), \fullcode\ = ?;
\item \book{Mots Codés} (\puzmagaz, \FRland).
\end{enumerate}
%
\item \barcodBa {1-453927-348790}. \barcodBe.
\item \dagblaNO {\textsl{Dagbladet}}. \gifulcod. \hvaderNO?
\by{---\HDname}
\GridEAN 0{111132111411123112311231123111113211213132112212212211231110}
\end{assgts}
%
\begin{figure}
\begin{enumerate}\renewcommand \labelenumi {\Alph {enumi}.}
\item\GridEAN 0{111321122121411222111231141111113211131212311312321121221110}\bigskip\smallskip
\item\GridEAN 0{111141112221312231111322212111113211141132113211113231121110}\bigskip\smallskip
\item\GridEAN 0{111131231211123123113212122111112221113231121231212212311110}\bigskip\smallskip
\item\GridEAN 0{111321111142311112312131222111111132321111322221111421221110}\bigskip\smallskip
\item\GridEAN 0{111113222121312211313213211111112122212232111312222112311110}\bigskip\smallskip
\item\GridEAN 0{111123111143121114111231114111113211321132113211131221221110}\bigskip\smallskip
\item\GridEAN 0{111141121132212321132111123111113211212211141114222121221110}\bigskip\smallskip
\item\GridEAN 0{111321112312113114132112131111112221321111321312113221221110}\bigskip\smallskip
\item\GridEAN 0{111113212221222221232113211111113211141111322122321131121110}
\end{enumerate}
\end{figure}
\pagebreak
\thispagestyle{answersh}
\mbox{}\vfill
\textbf{(b)} \GridEAN 1{111616161616161111111616161616161110}

\makepart{\solsindl}
\thispagestyle{empty}
\pagestyle{somestyle}
%
\solution \mezvestr:\bigskip\\
%
\begin{tabular}{|lc|}\hline
\textbf{ne-} & \mi \\
\textbf{ke-} & \yumi \\
\textbf{------} & \ta \\
\hline
\end{tabular}
\begin{tabular}{|ll|}\hline
\textbf{kaw} & \rootkaw \\
\textbf{ket} & \rootket \\
\textbf{kēsk} & \rootkEsk \\
\textbf{pahk} & \rootpahk \\
\textbf{pāhk} & \rootpAhk \\
\textbf{pīt} & \rootpIt \\
\textbf{taw} & \roottaw \\
\textbf{wack} & \rootwack \\
\textbf{wǣp} & \rootwAEp \\
\hline
\end{tabular}
\begin{tabular}{@{}r@{ }l@{}}
\multicolumn{2}{c}{\Vintrans:} \\
\begin{tabular}{|ll|}\hline
\textbf{-āhpe} & \suffAhpe \\
\textbf{-ānæhkæ} & \sAnaehkae \\
\textbf{-eqta} & ------ \\
\textbf{-ohnæ} & \suffohnae \\
\hline
\end{tabular} &
\begin{tabular}{|lc|}\hline
\textbf{-m} & \mi \\
\textbf{-q} & \yumi \\
\textbf{-w} & \ta \\
\hline
\end{tabular}\medskip \\
\multicolumn{2}{c}{\Vtransit:} \\
\begin{tabular}{|ll|}\hline
\textbf{-ah} & \mezah \\
\textbf{-aht} & \suffaht \\
\textbf{-en} & \mezen \\
\textbf{-es} & \suffes \\
\hline
\end{tabular} &
\begin{tabular}{|lc|}\hline
\textbf{-an} & \mi \\
\textbf{-æq} & \yumi \\
\textbf{-am} & \ta \\
\hline
\end{tabular}
\end{tabular}\medskip \\
%
\iambegin\ (\textbf e > \textbf{ǣ}).

\begin{assgts}
\item \begin{itemize}
\item \textbf{kekēskahæq}: \yumi\ \kekEskaeq {\mezah}
\item \textbf{nepāhkenan}:
\begin{itemize}
\item \mi\nepAhkarn {\mezen} ($\sqrt{\mbox{\tallstrut \textbf{pāhk}}}$),
\item \mi\nepAhkaln {\mezen} ($\sqrt{\mbox{\tallstrut \textbf{pahk}}}$)
\end{itemize}
\item \textbf{wǣpāhpew}: \ta\wAEpAhpew
\end{itemize}
\item \begin{itemize}\setlength{\itemsep}{0pt}
\item \mi\newAEpahtan: \textbf{newǣpahtan}
\item \yumi\ \kekAwaeq {\mezen}: \textbf{kekāwenæq}
\item \ta\tawAnaehkaew: \textbf{tawānæhkæw}
\item \ta\ketOhnaew: \textbf{ketōhnæw}
\end{itemize}
\end{assgts}

\solution \mbox{}\medskip \\
%
\begin{tabular}{p{142pt}p{320pt}}
\textbf{(a)} \begin{tabular}[t]{ll}
\farord{bøga} & \ipaword{bøːva} \\
\farord{knoðar} & \ipaword{knoːar} \\
\farord{kvøða} & \ipaword{kvøːa} \\
\farord{løgur} & \ipaword{løːvur} \\
\farord{plága} & \ipaword{plɔava} \\
\farord{skaði} & \ipaword{skɛaji} \\
\farord{toygur} & \ipaword{tɔijur} \\
\farord{trúgi} & \ipaword{trʉuwi} \\
\end{tabular} &
\textbf{(b)} \erstsyll\
\farord{a}~\ipaword{ɛa},
\farord{á}~\ipaword{ɔa},
\farord{e}~\ipaword{eː},
\farord{ei}~\ipaword{ai},
\farord{ey}~\ipaword{ɛi},
\farord{i}~\ipaword{iː},
\farord{o}~\ipaword{oː},
\farord{oy}~\ipaword{ɔi},
\farord{ó}~\ipaword{ɔu},
\farord{u}~\ipaword{uː},
\farord{ú}~\ipaword{ʉu},
\farord{ø}~\ipaword{øː}.\medskip

\bnvowels\ \farord{ð} = \farord{g}. \applerst:
\begin{enumerate}
\item\farord{ð/g} \ipaword{w} \textbar\ \ipaword{u(ː)} \underline{\quad};
\item\farord{ð/g} \ipaword{j} \textbar\ \ipaword{i(ː)} \underline{\quad} \au\ \underline{\quad} \ipaword{i(ː)};
\item\farord{ð/g} \ipaword{v} \textbar\ \underline{\quad} \ipaword{u(ː)};
\item\farord{ð/g} \ipaword{v} \inN, \ipaword{$\emptyset$} \inV.
\end{enumerate}
\end{tabular}

\solution \rulesmot:
%
\begin{enumerate}
\item \worderNA.
\item \Ngetmark {\orAhavan}{\vaiphra-},
\unlinali\ (\bodypart, \kinsterm);
\thenposs.
\item \alienpos {\bipa{á}}.
\item \incompoN\lastonlo\ (\textbf{\latehta\'}).
\end{enumerate}

\begin{assgts}
\item
%\begin{itemize}
%\item \vailien{mùsú}{gbòmù}: \musugbomu
%\item \vailien{léŋ kúndú}{nyìmìì}: \leNkundus {\nyimii}
%\item \vaiphra{gbòmù-l\'{ɛ}nd\'{ɛ} kúndú}: \gbomulEndEkundu
%\end{itemize}
\vailien{mùsú}{gbòmù}: \musugbomu \\
\vailien{léŋ kúndú}{nyìmìì}: \leNkundus {\nyimii} \\
\vaiphra{gbòmù-l\'{ɛ}nd\'{ɛ} kúndú}: \gbomulEndEkundu
\item \vaiphra{kánd\'{ɔ}-l\'{ɛ}nd\underline{\'{ɛ}} \lOO}: \kandOlEndElOO
\item
%\begin{itemize}
%\item \kOanjas {\nyimii}: \vailien{k\'{ɔ}ánjà}{nyìmìì}
%\item \leNlOOs {\ja}: \vainali{léŋ \lOO}{já}
%\item \kaijaNlOOmusu: \vainali{kàí jǎŋ}{\lOO-mùsù}
%\item \nyimiileNlOO: \vaiphra{nyìmìì-lèŋ \lOO}
%\end{itemize}
\kOanjas {\nyimii}: \vailien{k\'{ɔ}ánjà}{nyìmìì} \\
\leNlOOs {\ja}: \vainali{léŋ \lOO}{já} \\
\kaijaNlOOmusu: \vainali{kàí jǎŋ}{\lOO-mùsù} \\
\nyimiileNlOO: \vaiphra{nyìmìì-lèŋ \lOO}
\end{assgts}

\solution \incompoN\modihead.
\litleunl {\mbox{\bord{-tl/li}}}{\bord{-capil} (\dimin.), \bord{-huah} '\huah{\dots}',
\mbox{\bord{-tlah/lah}} '\tlah{\dots}'}{\au\ \bord{-tzintli} '\tzin{\dots}'}
(\afteroth{\bord{-li} \et\ \bord{-lah}}{\bord l}{\bord{-tl} \et~\bord{-tlah}}).
%
\begin{assgts}
\item\mbox{}
\begin{tabular}[t]{ll}
\bord{a-cal-huah} & \acalhuah\ (\bord{a-cal-li} \acalli, \quoted{\acalf}) \\
\bord{a-chil-li} & \achilli \\
\bord{a-tl} & \atl \\
\bord{cal-lah} & \callah \\
\bord{cal-huah} & \calhuah \\
\bord{chil-a-tl} & \chilatl \\
\bord{chil-li} & \chilli \\
\bord{col-li} & \collib/\collid \\
\bord{col-tzintli} & \tzin {\collib/\collid} \\
\bord{cone-huah} & \conehuah, \quoted{\conehuaf} \\
\bord{cone-huah-capil} & \conehuahcapil \\
\bord{cone-tl} & \conetl \\
\bord{oquich-cone-tl} & \oquichconetl, \oquichconeg \\
\bord{oquich-huah} & \oquichhuah, \quoted{\oquichhuaf} \\
\bord{oquich-totol-tzintli} & \tzin {\oquichtotolin} \\
\bord{te-huah} & \tehuag \\
\bord{te-tlah} & \tetlah \\
\bord{totol-te-tl} & \totoltetl \\
\end{tabular}
\item
\calli: \bord{calli} \hfill
\tetl: \bord{tetl} \hfill
\ahuah: \bord{ahuah} \\
\tzin {\oquichtl}: \bord{oquichtzintli}
\item
\bord{cacahua-tl}: \cacahuatl \hfill
\bord{cacahua-te-tl}: \cacahuatetl \\
\bord{cacahua-a-tl}: \cacahuaatl \hfill
\bord{cacahua-huah}: \cacahuahuah
\end{assgts}

\newpage
\mathchardef\black="020F
\mathchardef\white="020E
\solution
\barframe {$\black \white \black $}{$\white \black \white \black \white $}.
\digitcod {1--4}7.
\threecod R{A \et\ B}. \smallskip\\
%
\parbox{3.4in}{\subcodep AB.
\allstart A (\rightway, \elsemirr BR) \et\conthree A.
\feallbar {AABABB} (\subcod~1).}\hfill
\begin{tabular}{|l||c|r|r||r|}\hline
& & A: $\white \black \white \black$ & B: $\white \black \white \black$ & R: $\black \white \black \white$ \\\hline
\codeline 0{------}3211 \\
\codeline 1?2221 \\
\codeline 2{AABBAB}2122 \\
\codeline 3{AABBBA}1411 \\
\codeline 4{ABAABB}1132 \\
\codeline 5{ABBAAB}1231 \\
\codeline 6{ABBBAA}1114 \\
\codeline 7{ABABAB}1312 \\
\codeline 8{ABABBA}1213 \\
\codeline 9{ABBABA}3112 \\
X & AAABBB & ------ & ------ & ------ \\\hline
\end{tabular}\smallskip

\onpriceA\ (\onpriceB). \onpriceC {\subcod~2}, \onpriceD\ (\farsteak: 0416 $\to$ \euros4 \et\cents{16}).

\begin{assgts}
\item \begin{enumerate}
\item (E);
\item G, \checksum\ = 2;
\item C;
\item D;
\item A, \DEland;
\item I;
\item H, \cost\ = \euros4 \et\centa{74};
\item B, \fullcode\ = 7-317442-030049;
\item F.
\end{enumerate}
%
\item
\GridEAN 1{111113212311141311222122131111111411113212131312311232111110}
\hfill \barheight=40pt\EAN 1-453927-348790
\item \upsidown\ (\startwiB BA), \mustturn.

\NOland = 70, \fullcode\ = 7-022070-000035.
\hfill \barheight=40pt\EAN 7-022070-000035
\end{assgts}

\makepart{\probteam}
\thispagestyle{empty}
\pagestyle{somestyle}
%\section*{\disminut 0}
\introSkr. \werewell, \mumacrod, \vomacrob, \deschang, \onewhole. \sylstand\ (\sinodeld).

\sentowas 9{\bord{syād indravajrā yadi tau jagau ga\d{h}}}.
\canrecon {\bord a}{\bord{syād}}, \nonrecon {\bord a}{\bord{indravajrā}} (\orcompar {10}).
\fortpoet.

\longmark {\textbf{\latehta\=}, \camacron,};
%(\bord{ā}, \bord{ī}, \bord{ū}); \nomacurt.
\bord{b{\hh}}, \bord{d{\hh}}, \bord{g{\hh}}, \bord{\d{h}}, \bord j, \bord{ñ}, \bord{\d{n}}, \bord{ś}, \bord{t{\hh}} \et~\bord y \aconsons.
\yanother.

\trawrong.
\medskip \\
%
\begin{tabular}{rp{130pt}p{310pt}}
1.&\bord{b{\hh}ujanga-prayātam caturb{\hh}ir gakarai\d{h}} & \isquaple {\quoted{\bhujpray}}{\word{ga}}.\\
2.&\bord{gurunid{\hh}anamānulag{\hh}ur iha śāśikalā} & \ifatends {\word{guru}}{14}{\word{lag{\hh}u}}{\quoted{\CaCikalA}}.\\
3.&\bord{jarau jarau tato jagau ca pañcacamaram vadet}
& \An{\emph{ja}-\et-\emph{ra}}, \an{\emph{ja}-\et-\emph{ra}}\et\daaracht\an{\emph{ja}-\et-\emph{ga}}\vadethet {\quoted{\pancAmar}}.\\
4.&\bord{mab{\hh}alagā gajagati\d{h}} & \quoted{\gajagati}\is\ \emph{ma b{\hh}a la ga}.\\
5.&\bord{mo go go go vidyunmālā}
& \An{\emph{ma}}\et\an{\emph{ga}}\et\an{\emph{ga}}\et\an{\emph{ga}}\is\quoted{\vidymAlA}.\\
6.&\bord{nanagi mad{\hh}umati} & \whethere{\emph{na na ga}}{\quoted{\madhumat}}.\\
7.&\bord{prama\d{n}ikā \underline{\hspace*{0.4in}} \underline{\hspace*{0.4in}}} & \quoted{\pramANik}\is\ \underline{\hspace*{0.4in}} \underline{\hspace*{0.4in}}.\\
8.&\bord{pramā\d{n}ikā padadvayam vadanti pañcacāmaram} & \twolines {\quoted{\pramANix}}{\quoted{\pancAmar}}.\\
9.&\bord{syad indravājrā yadi tau \mbox{jagau} ga\d{h}} & \ifthrees{\emph{ta}}{\emph{ja}-\et-\emph{ga}}{\emph{ga}}{\quoted{\Indravaj}}.\\
10.&\bord{ūpendravajrā prat{\hh}ame lag{\hh}au sā} & \thinerst {\quoted{\Upendvaj}}{(\Indravaz)}{\emph{lag{\hh}u}}.\\
\end{tabular}

\begin{assgts}
\item \whataffd?
\item \redelmot, \remodlet, \redothem.
\item \offmacra {\bord{y\Katla m}}. \whatsyll {\emph{guru}}?
\end{assgts}
%%Yamātārājab{\hh}ānasalagam
%
\NB\quad
\mnemodef\ (\word{\pimnemon} $\mbox{\dots} \to 3\decpoint 14159 \approx \pi$).
\Upendrau.
\by{---\AHname}

\pagebreak

\section*{\disminun}
\follhint: \signifof {\emph{guru}}, \soonhint {\bord{y\Katla m}}.
\anshints {30}

\pagebreak

\section*{\disminut {30}}
\hintcomp {3 \et\ 8}{9 \et\ 10}. \zagrad{\justhint}

\defSguru {\word{guru}}. \divindiv. \sylladiv{VCV}{V-CV}{VCCV}{VC-CV}.

\thatlett {\bord a}\inSyll {12}\inLine 2 \addedlen.
%%%\pagebreak
\anshints {60}

\pagebreak

\section*{\disminut {60}}
\hintread. \zagrad{\justhint}

\selfdesc.

\lastlett {\bord g}\inLine 1 \isbroken.
%%%\pagebreak
\anshints{90}

\pagebreak

\section*{\disminut {90}}
\kendinna, \resthint{\bord{ta}}{\bord{ra}}. \zagrad{\justhint}

\descmetr {\bord{y\Katla (m)}}.

\thatlett {\bord a}\inSyll 2 \inLine 7 \deledlen.
%%%\pagebreak
\anshints {120}

\pagebreak

\section*{\disminut {120}}
\quadusef 1. \zagrad{\justhint}

\Twosylls\ \bord{la} \et\ \bord{ga} \standfor {\bord{lag{\hh}u} \et\ \bord{guru}}.
\mnemexpl {\bord{y\Katla (m)}}{\word{lag{\hh}u} \au\ \word{guru}}. 

\thingist {\bord e \et\ \bord o}, \barnomac.
\zagrad{\infacteo {\bord{ai} \et\ \bord{au}}{\bord{āi} \et~\bord{āu}}}
%%%\pagebreak
\anshints {150}

\pagebreak

\section*{\disminut {150}}
\poetinte. \eginluna {14}. \zagrad{\justhint, \nonguess}

\mnemexpn 8{\bord{y\Katla m}}{\emph{lag{\hh}u} \et\ \emph{guru}}, \macresto.

\adelword\ \bord{jarau}.
%%%\pagebreak
\anshints {180}

\pagebreak

\makepart{\soluteam}
\thispagestyle{empty}
\pagestyle{somestyle}
\section*{\disminut {180}}
\defSguru {\word{guru}}. \divindiv. \sylladiv{VCV}{V-CV}{VCCV}{VC-CV}.

\selfdesc.

\mnemexpn 8{\bord{yamātārājab{\hh}ānasalagam}}{\emph{lag{\hh}u} \et\ \emph{guru}}.
\Twosylls\ \bord{la} \et\ \bord{ga} \standfor {1~\word{lag{\hh}u} \et\ 1~\word{guru}}.

\makemnem, \tritridu {\bord{la} \au\ \bord{ga}}.

\begin{assgts}
\item \whataffd?\par \respaffd.
\item \begin{itemize}
\item \Delwords: \bord{jarau}, \bord{lagau}.
\item \dLetters:
\begin{enumerate}
\item \Lineword~1: \word{b{\hh}ujanga-prayātam caturb{\hh}ir\textbf gakarai\d{h}} < \word{b{\hh}ujanga-prayātam caturb{\hh}ir\textbf yakārai\d{h}}
\item \Lineword~4: \word{\textbf mab{\hh}alagā gajagati\d{h}} < \word{\textbf nab{\hh}alagā gajagati\d{h}}
\item \Lineword~5: \word{mo \textbf go go go vidyunmālā} < \word{mo \textbf mo go go vidyunmālā}
\end{enumerate}
\item \Addmacra:
\begin{enumerate}
\item \Lineword~2: \word{gurunid{\hh}anam\textbf{ā}nulag{\hh}ur iha śāśikalā} < \word{gurunid{\hh}anam\textbf anulag{\hh}ur iha śaśikalā}
\item \Lineword~2: \word{gurunid{\hh}anamānulag{\hh}ur iha ś\textbf{ā}śikalā} < \word{gurunid{\hh}anamanulag{\hh}ur iha ś\textbf aśikalā}
\item \Lineword~9: \word{syad indrav\textbf{ā}jrā yadi tau \mbox{jagau} ga\d{h}} < \word{syād indrav\textbf ajrā yadi tau jagau ga\d{h}}
\item \Lineword~10: \word{\textbf{ū}pendravajrā prat{\hh}ame lag{\hh}au sā} < \word{\textbf upendravajrā prat{\hh}ame lag{\hh}au sā}
\end{enumerate}
\item \Delmacra:
\begin{enumerate}
\item \Lineword~1: \word{b{\hh}ujanga-prayātam caturb{\hh}ir gak\textbf arai\d{h}} < \word{b{\hh}ujanga-prayātam caturb{\hh}iryak\textbf{ā}rai\d{h}}
\item \Lineword~3: \word{jarau jarau tato jagau ca pañcac\textbf amaram vadet} < \word{\dots\ pañcac\textbf{ā}maram vadet}
\item \Lineword~6: \word{nanagi mad{\hh}umat\textbf i} < \word{nanagi mad{\hh}umat\textbf{ī}}
\item \Lineword~7: \word{pram\textbf a\d{n}ikā \underline{\hspace*{0.4in}} \underline{\hspace*{0.4in}}} < \word{pram\textbf{ā}\d{n}ikā jarau lagau}
\item \Lineword~9: \word{sy\textbf ad indravājrā yadi tau \mbox{jagau} ga\d{h}} < \word{sy\textbf{ā}d indravajrā yadi tau jagau ga\d{h}}
\end{enumerate}
\end{itemize}
\item \Syllsare {2, 3, 4, 6}{10}{\word{guru}}: \bord{yamātārājab{\hh}ānasalagam}.
\end{assgts}

\end{document}

\pagestyle{empty}
%\section*{\teamiont\ (\razdaWed)}
\makepart{\teamiont}
\teamiona {\quoted{\disminun. \colminut{30}}}.

\teamionb {\ansinsix, \nodyssey}\iconvert.
\teamionc {\quoted{\disminut {30}. \colminut{60}}}, \teamionz.
\teamiond {60, 90, 120, 150}{180}, \barfinal {180}.

\teamione: \teamionf {10\%}.
\teamiong\ (\latignor). \indepont, \teamionh. \zagrad{\teampteg, \teamioni}

\teamionj.
\teamionk.

\teamionl, \makenots.
\zagrad{\teamionm, \teamionn}
%(\teamionn.)

\end{document}
