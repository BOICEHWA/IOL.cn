\def \thisling{中文}
\def \thislang{中文}
\def \thistext{中文文本}
\def \olympiad{第八届国际语言学奥林匹克竞赛}
\def \xcountry{瑞典}
\def \yvillage{斯德哥尔摩}
\def \Julyname{7月}
\def \olydates #1#2#3#4{#4年#3#1 – #2日}
\def \probindl{个人赛题目}
\def \solsindl{个人赛解答}
\def \probteam{团体赛题目}
\def \soluteam{团体赛解答}
\def \probword #1{题 \##1}
\def \pontword{分}
\def \regulats{解答规则}
\def \regulatx{毋需抄题. 将不同问题的解答分述于不同的答题纸上. 每张纸上注明题号、座位号和姓名}
\def \towarrant{否则答题纸可能被误放或遗失.}
\def \regulaty{解答需详细论证. 无解释之答案, 即便完全正确, 也会被处以低分.}
\def \editorsz{编者}
\def \edinchef{主编}
\def \goodluck{祝你好运}
\def \rulesmot{规则}
\def \answersp{答案}
\def \gisentra #1{下列是#1语句及其汉语翻译}
\def \andtrans{及其汉语翻译}
\def \chaotict{(乱序排列)}
\def \fordinte #1{翻译称#1}
\def \fordinto #1{翻译成#1}
\def \corrcorr{请将其予以对应}
\def \fillgaps{填补空缺}
\def \filmties{填补空缺}
\def \neshaded{以阴影表示者毋需填充}
\def \onBliss #1{布里斯符号是奥地利裔澳大利亚人查尔斯 · K. 布里斯 (#1) 发明的一套通用符号系统. 布里斯认为, 该符号系统可为任意母语的人所理解}
\def \inBliss{下列是用布里斯符号书写的词语}
\def \xieBliss{用布里斯符号表示}
\def \indimean{表明以下符号的意义}
\def \knowbiid{已知其中2个符号意义相同}
\def \posleft{若词语中包含不止1个符号, 该表记置于最左侧符号之上方}
\def \posX{词类}
\def \posN{名词}
\def \posA{形容词}
\def \posV{动词}
\def \composit{组合}
\def \BUGmeaning{意义}
\def \isavowel{是元音}
\def \aconsons{是辅音}
\def \owumlaut{= 汉语拼音 \word{üe}}
\def \diumlaut{= 汉语拼音 \word{üe} 与 \word{ü}}
\def \narschwa{$\approx$ 英语 \word{but}中的\word{u}}
\def \jotsound #1{#1~= 普通话 \word{叶} (\word{yè})的 \word{y}}
\def \retrflet #1{#1 $\approx$ 通用北美口音英语 \word{art} 中的 \word{t}, 发音时舌尖向后卷}
\def \velarnas #1{#1~= 普通话 \word{杭} (\word{háng})中的\word{ng}}
\def \chaffric #1{#1 $\approx$~普通话\word{七} (\word{qī})中的\word{q}}
\def \hwisound #1{#1 是汉语拼音 \word{w} 的清音 (如苏格兰或南部美国口音 \word{which} 中的 \word{wh})}
\def \infamily #1#2{#1隶属于#2语系}
\def \lgBudukh{布都赫语}
\def \lgDrehu{利富语}
\def \toSECauc{东北高加索}
\def \toAustNs{南岛}
\def \spokenca #1#2{#2, 约有\ #1人使用该语言}
\def \inAzerba{在阿塞拜疆}
\def \inGrauCH{在格劳宾登州}
\def \isleLifu{在新喀里多尼亚以东的利福岛}
\def \rhaetoro{罗曼什语隶属于罗曼语族的列托-罗曼斯亚族}
\def \widefrit{该语言同德语、法语和意大利语并为瑞士的4种国语}
\def \inDrehu{下列是部分利富语数词 (字母顺序排列) 及其表示的数值 (升序排列)}
\def \writnums{用阿拉伯数字表示}
\def \writDreh{翻译成利富语}
\def \introBud{下列是布都赫语动词的3种形式}
\def \futtense{将来时}
\def \prohimod{禁止语气}
\def \nthclass #1{第#1类}
\def \masculin{阳性}
\def \feminine{阴性}
\def \donVnext #1{其中 #1 取决于下一音节之元音}
\def \ifendsin #1{若词干以#1结尾}
\def \wherRdef #1#2#3{若#2包含于词根中, 则#1为该流音; 否则, #1是#3}
\def \unRfollo #1{除非#1恰附于其后}
\def \with #1{中#1}
\def \before #1{在#1前}
\def \anotherC{另一个辅音}
\def \wiotherC #1{#1和另一个辅音的组合}
\def \unotherC{不含另一个辅音}
\def \postVone{第1个元音后}
\def \dialname #1{#1}
\def \Ssil{苏斯勒万方言}
\def \Edin{恩嘎丁方言}
\def \inRomansh{下列是罗曼什语2种方言的部分词语}
\def \somegaps{部分单词已被略去}
\def \whatisin #1#2{#2在#1中怎么说}
\def \etwhatin #1{在#1中呢}
\def \plursare #1#2#3#4#5{在#1中, #2是#4, #3是#5}
\def \plurelse #1#2{你可能认为这2个词在#1中说法相同, 但实际上它们分别是 #2}
\def \howexpla{如何解释}
\def \tranboth{分别翻译成这2种方言}
\def \bothdial{(2种方言相同)}
\def \brokrule #1#2{在#1中, (与#2不同) 第1条规则对复数形式并不适用}
\def \hypfirst{换言之, 当一个辅音位于词干中, 而另一个辅音位于后缀中时}
\def \hyptwond{或当元音在后缀添加前已经被选定时}
\def \hypthird{或当复数中的元音需与单数中的元音匹配时, 该规则不适用}
\def \inGenCod{遗传学的一项伟大成就是对基因密码的破解——mRNA-多肽词典的创建}
\def \proteins{多肽 (蛋白质) 是构造所有生物的“砖块”}
\def \aminacid #1{多肽分子是链状结构, 由氨基酸 (用诸如#1的符号表示) 组成. 多肽的性质由其氨基酸序列所决定}
\def \nucletid #1{细胞合成多肽时, 听从信使核糖核酸 (mRNA) 分子上书写的指令, mRNA 链包含4种核苷酸 (分别用#1表示)}
\def \ifuseRNA{如果一个细胞用以下的mRNA序列作为模板}
\def \synthese{则以下的多肽将被合成}
\def \anuseRNA{一个细胞用以下的mRNA序列作为模板}
\def \syntwhat{则它将合成何种多肽}
\def \synthest{某细胞合成了如下的多肽}
\def \usedwhat{则它可能使用了何种mRNA序列作为模板}
\def \rootypes{核苷酸对有时称作 \textbf{根} , 并可分为2类}
\def \sicroots #1{#1根}
\def \forPlNom{强}
\def \debPlNom{弱}
\def \egroots #1#2{#1根的例子有#2}
\def \forPlGen{强}
\def \debPlGen{弱}
\def \definfor #1#2{若#2编码同样的氨基酸, 则#1为强根}
\def \defindeb{否则, 其为弱根}
\def \claroots{将其它的根予以分类}
\def \mRNAwith #1{所有的 mRNA 序列皆由 #1 起始}
\def \polycont #1{题例中的4个多肽链由 #1 个氨基酸组成}
\def \fragcont #1{而 mRNA 序列包含了 #1 个核苷酸}
\def \trioprob{由此可知, 一组3个核苷酸或表记1个氨基酸, 或分隔不同的多肽 (实际上是终止多肽合成的信号)}
\def \butalloc #1#2{然而, 一组3个核苷酸有 #1 种可能的组合方式 (除2个外, 题例全部涉及), 却仅有 #2 种不同的氨基酸. 因此, 某些不同的核苷酸组合可能表示相同的意义}
\def \waitillc{该序列包含题例中缺失的2组核苷酸组合, 因而暂时无法确定答案. 但当本题解答完整后, 此处的疑问将得以确认}
\def \simpdata{本题呈现的数据进行了适当简化}
\def \possties{种可能}
\def \pointuse #1{指事符 #1 用以指示符号的某一特定部分}
\def \pointer{指事符}
\def \pointers{指事符}
\def \surwards{上部}
\def \subwards{下部}
\def \outwards{外部}
\def \inMongol #1{考虑下列选自蒙古语单语词典 (#1) 的单词及其释义 (用拉丁字母转写表示)}
\def \tranmost{尽可能多地翻译文中的蒙古语单词}
\def \timeword{时间}
\def \stomach{胃}
\def \woman{女人}
\def \world{世界}
\def \nummer{数}
\def \montpass{隘口}
\def \filling{牙齿填充物}
\def \mashnoun{浆糊}
\def \puolpa{干肉}
\def \fuorma{形式}
\def \labour{劳动}
\def \shortmot{短的}
\def \coulant{慷慨的}
\def \showverb{展示}
\def \totall{全部的}
\def \finally{最终}
\def \discuors{对话}
\def \elmsing{榆树}
\def \namesing{名字}
\def \ugolsing{角}
\def \florsing{花}
\def \parents{父母}
\def \elmplur{榆树(复数)}
\def \nameplur{名字(复数)}
\def \ugolplur{角(复数)}
\def \florplur{花(复数)}
\def \overtake{占领}
\def \alqol{坐下}
\def \arxar{睡觉}
\def \qalqal{倚靠}
\def \sonkon{吃惊}
\def \ynkan{保持}
\def \jechi{穿过}
\def \woltu{捆, 系}
\def \halghu{吞咽}
\def \harki{袭击 (动物)}
\def \horchu{推}
\def \jolku{滚}
\def \quroghu{暂停}
\def \osu{放置}
\def \chighi{推动}
\def \chorhucu{交换}
\def \ensi{熄灭}
\def \liquid{液体}
\def \water{水}
\def \airstuff{空气}
\def \gasstuff{气体}
\def \fogstuff{雾}
\def \bigcircS{圆圈}
\def \aurinkoS{太阳}
\def \headneck{头和颈}
\def \eyezbrow{眼睛和眉毛}
\def \bodytors{躯干}
\def \waist{腰}
\def \heart{心}
\def \necknoun{颈}
\def \eyebrow{眉毛}
\def \eyenoun{眼睛}
\def \nosenoun{鼻子}
\def \mouth{嘴巴}
\def \lipsnoun{双唇}
\def \saliva{口水}
\def \eastside{东方}
\def \western{西方的}
\def \merry{快乐的}
\def \malmerry{沮丧的}
\def \sickill{病的}
\def \blowverb{吹}
\def \breathe{呼吸}
\def \riseverb{升起}
\def \cryweep{哭}
\def \activity{活动性}
\def \activadj{活动的}
\def \activerb{活动}
\def \ABname{Alexander Berdichevsky}
\def \AHname{Adam Hesterberg}
\def \ANname{Aleksei Nazarov}
\def \APname{Alexander Piperski}
\def \BBname{Bozhidar Bozhanov}
\def \BIname{Boris Iomdin}
\def \DGname{Dmitry Gerasimov}
\def \IDname{戴谊凡}
\def \LFname{Ludmilla Fedorova}
\def \MRname{Maria Rubinstein}
\def \RPname{Renate Pajusalu}
\def \SBname{Svetlana Burlak}
\def \SGname{Stanislav Gurevich}
\def \TTname{Todor Tchervenkov}
\def \XGname{Ksenia Gilyarova}
\def \QCname{曹起曈}
\def \MLname{刘闽晟}
\def \edinames{\ABname, \BBname, \SBname, \IDname, \LFname, \DGname, \XGname, \SGname, \AHname, \BIname, \ANname, \RPname, \APname\ (\edinchef), \MRname, \TTname}
\def \quoted #1{“#1”}
\def \enqueten{姓名}
\def \enquetep{座位号}
\def \enquetea{你做了哪些题目?}
\def \enqueteb{你最喜欢哪道题?}
\def \enquetec{你觉得哪道题最难?}
\def \enqueted{你觉得哪道题最简单?}
\def \et{与}
\def \ab{或}
\def \au{或}
\def \whowroti{\QCname, \MLname}
\def \whowrotj{\BIname}
