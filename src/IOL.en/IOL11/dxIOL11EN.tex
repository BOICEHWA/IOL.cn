\def \thistext{English text}
\def \nthIOL #1{#1 International Olympiad in Linguistics}
\def \thisth{Eleventh}
\def \thislandd{United Kingdom of Great Britain and Northern Ireland}
\def \thisland{Great Britain}
\def \thistown{Manchester}
\def \Julyname{July}
\def \Auguname{August}
\def \olydates #1#2#3#4{#1–#2 #3 #4}
\def \olydatez #1#2#3#4#5{#1~#2 – #3 #4 #5}
\def \leafword #1{Sheet \##1}
\def \probword #1{Problem \##1}
\def \probsing #1{#1 Problem}
\def \probplur #1{#1 Problems}
\def \answersp{Answers}
\def \respsing #1{#1 Answers}
\def \solusing #1{#1 Solution}
\def \soluplur #1{#1 Solutions}
\def \indicont{Individual Contest}
\def \teamcont{Team Contest}
\def \Teamword{Team}
\def \pontword{points}
\def \regulats{Rules for writing out the solutions}
\def \regulado{Do not copy the statements of the problems}
\def \regulare{Write down your solution to each problem on a separate sheet or sheets}
\def \regulami{On each sheet indicate the number of the problem, the number of your seat and your surname}
\def \towarrant{Otherwise your work may be mislaid or misattributed}
\def \regulaty{Your answers must be well-argumented}
\def \regulatz{Even a perfectly correct answer will be given a low score unless accompanied by an explanation}
\def \editorsz{Editors}
\def \edinchef{editor-in-chief}
\def \rulesmot{Rules}
\def \enquetex{Questionnaire}
\def \enqueten{Name}
\def \enquetep{Place number}
\def \enquetea{Which problems did you work on}
\def \enqueteb{Which problem did you like best}
\def \enquetec #1{Which problem did you find #1}
\def \seemhard{hardest}
\def \seemeasy{easiest}
\def \goodluck{Good luck}
\def \quoted #1{“#1”}
\def \transcri{Write down how the following words and phrases are pronounced}
\def \giwordss #1{Here are some words and phrases in #1}
\def \givemots #1{Here are some words in #1}
\def \givesent #1{Here are some sentences in #1}
\def \inthelg #1{#1}
\def \Muna{Muna}
\def \Piraha{Pirahã}
\def \Yidiny{Yidiny}
\def \Yukaghir{Yukaghir}
\def \inTundra #1{Tundra #1}
\def \andtrans #1{and their #1 translations}
\def \siscript{simplified transcriptions of their pronunciation in normal speech}
\def \chaotict{in arbitrary order}
\def \madelist #1{Fifteen years ago the British poet, critic, and biographer #1 compiled a certain list}
\def \ginNuskh #1{The following two pages contain that list in a Georgian translation, written in the ancient Nuskhuri alphabet (#1th century)}
\def \tranmost #1{Translate as much of it as you can into #1}
\def \fordinto #1{Translate into #1}
\def \tothislang{English}
\def \tolgsky #1{#1}
\def \tolg #1{#1}
\def \Thelg #1{The #1 language}
\def \infamily #1#2{#1 belongs to the #2 family}
\def \toAustNs{Austronesian}
\def \toPNyfam{Pama–Nyungan}
\def \spokenca #1#2{It is spoken by approx.\ #1 people #2}
\def \inInesia{in Indonesia}
\def \inQueend{in the state of Queensland, Australia}
\def \piragena #1{#1 is the indigenous language of the isolated #1 people of Amazonas, Brazil}
\def \soloMura{It is the only surviving member of the Mura language family}
\def \onyukaga #1{Tundra #1 is the language of a small population in Northeast Siberia}
\def \onyukagb{It is only spoken by several dozen mostly aged people}
\def \onyukagc #1#2{as many #1s have abandoned it in favour of Russian or one of the languages of their more numerous neighbours, who are often bearers of more advanced material cultures #2}
\def \egYakut{such as the Yakuts}
\def \ifnosure{If you aren’t sure how to translate some word, explain why}
\def \motmeans #1#2{The word #1 means #2}
\def \twomsame{Two of these words have the same meanings as two of the words in the data}
\def \CMUniloc{Carnegie Mellon University}
\def \countUSA{USA}
\def \villPitt{Pittsburgh}
\def \telepata #1#2{In a series of experiments run in #2 in #1, volunteers were first shown some English words, while activity was being registered in different locations of their brains}
\def \location #1{location #1}
\def \activite #1#2{Location #1 #2}
\def \activaby #1{is activated by #1}
\def \longmots{long words}
\def \actibyth #1{\activaby {the idea of #1}}
\def \shelterO{shelter}
\def \shelter{shelter}
\def \manipulO{manipulation}
\def \manipul{manipulation}
\def \feedingO{eating}
\def \feeding{eating}
\def \high{high}
\def \low{low}
\def \telepatc #1{Then the volunteers were asked to think of some other words from a preselected list of #1 words, while the researchers were measuring their brain activity again}
\def \telepats #1{The researchers claim that the first three factors have high ecological validity #1 and survival value}
\def \telepati{i.~e., the results of the experiment conform to the data on human behaviour in real life}
\def \telepatt #1{and that Location #1 is responsible for a low-level visual representation of the printed word}
\def \telepate{Below you can find some data on the activity levels for four brain locations depending on which word the volunteers were thinking of}
\def \telepatd{Using the obtained data, the researchers were able to determine the words the volunteers were thinking of quite successfully}
\def \sameinfo{The same information is given below on six more words the volunteers were thinking of}
\def \explasol{Explain your solution}
\def \corrcorr{Determine the correct correspondences}
\def \litt{lit.}
\def \joqol{Yakut person}
\def \joqon #1{Yakut #10}
\def \Aj #1{\relax}
\def \saan #1{wooden #10}
\def \alter #1{another #10}
\def \xogiaI{big}
\def \xogiai #1{big #10}
\def \toio #1{old #10}
\def \xitaixi #1{heavy #10}
\def \baihiigi #1{slow #10}
\def \xaibogi #1{fast #10}
\def \sabi #1{angry #10}
\def \hoigi #1{dirty #10}
\def \genitive{genitive}
\def \nnigen #1{\nni{#1}}
\def \nni #1{of \an #10}
\def \mmu #1{from \an #10}
\def \ggu #1{for \an #10}
\def \mujay #1{with \an #10}
\def \gimbal #1{without \an #10}
\def \waval #1{boomerang}
\def \mujam #1{mother}
\def \mugaRu #1{fishing net}
\def \majur #1{frog}
\def \bunya #1{woman}
\def \Man #1{man}
\def \mulari #1{initiated man}
\def \gajagimba #1{white man}
\def \bama #1{person}
\def \judulu #1{pigeon}
\def \muyubara #1{stranger}
\def \bimbi #1{father}
\def \galika #1{dog}
\def \binyjin #1{hornet}
\def \bajigal #1{tortoise}
\def \uo{child}
\def \uoduo{grandchild}
\def \aariinmyver{gunshot}
\def \aariismyver{rifle thunder}
\def \myver{thunder}
\def \saal{wood}
\def \aarii{rifle}
\def \aariinjohul{rifle’s muzzle}
\def \aariidovoj{rifle case}
\def \johudawur{nose case}
\def \yohudawur{A nose case serves to protect the nose from the cold}
\def \johul{nose}
\def \uodawur{cradle}
\def \ewceB{point}
\def \ewceA{tip}
\def \cohojedewce{tip of knife’s blade}
\def \johudewce{tip of nose}
\def \joqodile{horse}
\def \ile #1{deer}
\def \ilennime{herd of deer}
\def \ilesnime{house of deer}
\def \ilenlegul{deer feed}
\def \legul{food}
\def \legudovoj{sack for provisions}
\def \ovoj{bag}
\def \cuo{iron}
\def \cuon #1{iron #1}
\def \cireme{bird}
\def \ciremennime{nest}
\def \johunmyver{snoring}
\def \airplane{airplane}
\def \apartment{apartment}
\def \corn{corn}
\def \lettuce{lettuce}
\def \cup{cup}
\def \igloo{igloo}
\def \key{key}
\def \screwdriver{screwdriver}
\def \bed{bed}
\def \butterfly{butterfly}
\def \cat{cat}
\def \cow{cow}
\def \abeillex{bee}
\def \refrigerator{refrigerator}
\def \spoon{spoon}
\def \wordword{word}
\def \transion{translation}
\def \undernom{The underlined names belong to characters in stories}
\def \adhiadhini{The \underline{Demon}}
\def \dhini{The demon}
\def \dhinihi{The demons}
\def \lambuku{My house}
\def \munasubc{The ants’ houses}
\def \lagahi{The ants}
\def \lagahino{His ants}
\def \alaalaga{The \underline{Ant}}
\def \munasubj{My pupil’s ants}
\def \murihi{The pupils}
\def \murihino #1{The #1’s pupils}
\def \molohino #1{The #1’s mountains}
\def \munasubn{My women’s monkeys}
\def \robhinehi{The women}
\def \robhineno{His woman}
\def \kontuhi{The stones}
\def \damanaka{will be warm}
\def \dofanaka{are warm}
\def \dokodoho{are far}
\def \nakudomoho{will be far}
\def \munazina{are climbing the mountain}
\def \munazinm{are going to the \underline{Demon}}
\def \munazini{are returning to the houses}
\def \munazink{is cutting their mountain}
\def \munazinp{is buying my ants}
\def \munazine{is eating my monkeys}
\def \munazinf{will buy the \underline{Demon}’s house}
\def \munazinl{will climb the pupil’s stone}
\def \munazinn{will cut my bananas}
\def \munazind{will eat the woman’s banana}
\def \munazinq{will return to the women’s pupil}
\def \munazinh{will go to the \underline{Monkey}}
\def \andoandoke{The \underline{Monkey}}
\def \demonkey{monkey}
\def \monkey #1{monkey}
\def \nime #1{house}
\def \cohoje #1{knife}
\def \saadovoj #1{box}
\def \bahoigatoi #1{pig}
\def \hixi #1{rat}
\def \kagahoaogii #1{papaya}
\def \kahai #1{arrow}
\def \kaoaibogi #1{jungle spirit}
\def \poogaihiai #1{banana}
\def \xabagi #1{toucan}
\def \jaguar #1{jaguar}
\def \baagiso #1{many #11s}
\def \bagiaibaabi{bad thief}
\def \bigy{ground}
\def \kagihi{wasp}
\def \kapiigaiitoii{pencil}
\def \koxopa{stomach}
\def \piahaogixisoaipi{banana for cooking}
\def \tagasaga{machete}
\def \xaaibi{thin}
\def \xiga{hard}
\def \xaapisi{arm}
\def \xiiaapisi{sleeve}
\def \xisipoai{wing}
\def \xaogii{foreign woman}
\def \xibogi{milk}
\def \xisitaixagai{crooked feather}
\def \xagai{crooked}
\def \xisoobai{otter}
\def \xitiixisi{fish}
\def \giisai{this}
\def \persnumb #1#2{#1\ifcase #1\or st\or nd\or rd\fi\ person #2}
\def \Sg{sg}
\def \Pl{pl}
\def \singular{singular}
\def \plural{plural}
\def \Anim{anim}
\def \Inan{inan}
\def \Prs{present}
\def \Fut{future}
\def \carrepla #1#2{if the first sound of the root is #1, it is replaced by #2}
\def \elsinfix #1{otherwise #1 is inserted after the first consonant}
\def \directiP #1{The preposition #1 indicates the direction of motion}
\def \attribum{modified}
\def \attribut{modifier}
\def \pronoone{A noun and a following modifier are pronounced as one word}
\def \butcross #1#2{but at the end of the first word of the phrase #1 after a vowel is lost and in the beginning of the second word #2 is lost}
\def \fromwend{The syllabification starts from the end of the word}
\def \Sywexier{Syllable weight hierarchy}
\def \primrule{The rightmost syllable of the heaviest type among the last three syllables of the word receives primary stress}
\def \secorula{A phrase has a secondary stress if the last three syllables of the phrase don't contain any part of the first word}
\def \secorule{It is placed according to the same rules as the primary stress, but disregarding the last three syllables}
\def \iftotpar #1#2{if the number of syllables in the #1 is #2}
\def \wordLoc{word}
\def \stemLoc{stem}
\def \stemNom{stem}
\def \radical{root}
\def \radicar{the first root syllable}
\def \endingN{ending}
\def \allshort{all syllables are short}
\def \sodo{even}
\def \liho{odd}
\def \lgulpart{the last even-numbered syllable of the stem is lengthened}
\def \thatleng #1{the #1 syllable of the stem is lengthened}
\def \ultimate{final}
\def \penultim{penultimate}
\def \folloses #1{If the ending #1 immediately follows a long vowel, it loses its own vowel}
\def \wordsord{The word order is}
\def \sestruct{The sentences have the following structure}
\def \structNN{The compound noun has the following structure}
\def \Propname{Proper name}
\def \nounword{noun}
\def \adjectif{adjective}
\def \verbword{verb}
\def \Pred{predicate}
\def \Sb{subject}
\def \Ob{object}
\def \Posson{Possession}
\def \Possor{possessor}
\def \Possum{possessed}
\def \artimuna #1{If the subject is neither possessed nor a proper name, it is preceded by the article #1}
\def \simotend #1{if the word ends in a #1}
\def \byavowel{\vocal}
\def \byacsant{\const}
\def \vocal{vowel}
\def \const{consonant}
\def \prevocal{before a vowel}
\def \preconst{before a consonant}
\def \voiced #1{voiced #1}
\def \unvoiced #1{voiceless #1}
\def \isaconst{is a consonant}
\def \aconsons{are consonants}
\def \syllable{syllable}
\def \twosylla{two syllables}
\def \trosylla{more than two syllables}
\def \wordlast{the final sound of the word}
\def \islost{is lost}
\def \glotfric{\word h in English \word{hat}}
\def \glotstop{known as the glottal stop}
\def \hetteken #1{The mark~#1}
\def \markslen{denotes vowel length}
\def \marpslen{indicates that the preceding vowel is long}
\def \syllbond{shows syllable boundaries}
\def \stremark #1{before a syllable indicates #1 stress}
\def \primary{primary}
\def \secondary{secondary}
\def \ifthEone{if there is one}
\def \ifthEany{if any}
\def \marklong{Mark the long vowels}
\def \et #1{and #1}
\def \ab{or}
\def \au{or}
\def \like{as}
\def \APname{Alexander Piperski}
\def \BIname{Boris Iomdin}
\def \DGname{Dmitry Gerasimov}
\def \BBname{Bozhidar Bozhanov}
\def \TTname{Todor Tchervenkov}
\def \IDname{Ivan Derzhanski}
\def \PSname{Pavel Sofroniev}
\def \XGname{Ksenia Gilyarova}
\def \SGname{Stanislav Gurevich}
\def \LFname{Liudmila Fedorova}
\def \SBname{Svetlana Burlak}
\def \MRname{Maria Rubinstein}
\def \ABname{Aleksandrs Berdičevskis}
\def \LPname{Aleksejs Peguševs}
\def \ASname{Artūrs Semeņuks}
\def \DRname{Daniel Rucki}
\def \MSSname{Martin Seymour-Smith}
\def \BLname{Bruno L’Astorina}
\def \HDname{Hugh Dobbs}
\def \GHname{Gabrijela Hladnik}
\def \RSname{Rosina Savisaar}
\def \JLname{Jae Kyu Lee}
\def \edinames{\SBname, \IDname, \HDname, \DGname, \XGname, \SGname\ \zagrad {\edinchef}, \GHname, \BIname, \BLname, \JLname, \LPname, \MRname, \DRname, \RSname, \ASname, \PSname, \TTname}
\def \whowroti{\IDname, \XGname, \BIname, \ASname}
